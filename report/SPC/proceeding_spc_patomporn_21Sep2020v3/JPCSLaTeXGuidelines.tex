\documentclass[a4paper]{jpconf}
% \usepackage[square,sort&compress,sectionbib]{natbib}

\usepackage{amsmath}
\usepackage{graphicx}
\usepackage{wrapfig}
\usepackage{cite}
\usepackage{placeins}
\usepackage[table,xcdraw]{xcolor}


% define color
\definecolor{olivegreen}{RGB}{34,139,34}
\definecolor{darkorange}{RGB}{255,140,0}

\begin{document}
\title{Preliminary indirect measurement of cosmic-ray proton spectrum using Earth's $\gamma$-ray data from {\it Fermi} Large Area Telescope}

\author{Patomporn Payoungkhamdee}
\address{Department of Physics, Faculty of Science, Mahidol University, Bangkok 10400, Thailand}
% \address{Production Editor, \jpcs, \iopp, Dirac House, Temple Back, Bristol BS1~6BE, UK}

\ead{patomporn.pay@gmail.com}

\begin{abstract}
Cosmic rays (CRs) are high-energy particles, mostly protons, propagating in space. The rigidity (momentum per charge) spectrum of CRs is well described by a power law for which the spectral index is approximately 2.8 around 30 - 1000 GV. Recent measurements by PAMELA and AMS-02 indicate an abrupt change of the CR proton spectral index at about 340 GV. When CRs interact with the Earth's upper atmosphere, $\gamma$ rays can be produced and detected by space-based detectors. Here we use the Earth's $\gamma$-ray data collected by the {\it Fermi} Large Area Telescope along with a proton-air interaction model to indirectly determine the CR proton spectral index and compare against observations by other instruments.
\end{abstract}

\section{Introduction}
Cosmic-ray are high energy particle which mainly come from the outer space which can penetrate and interact with the Earth's atmosphere \cite{HESS,Pacini,Clay}.
The shark peak of gamma-ray emission from Earth's limb are mainly come from the interaction of CRs with the atmospheric molecules \cite{Warit2009}.

There are many possible phenomena of acceleration mechanism in the
space that could produce high energy particles. The characteristic of acceleration mechanism could roughly be distinguished by a spectral index in the arrival of cosmic rays spectrum in rigidity.
The breaking point of the spectrum mainly come from the overlapped region of acceleration mechanism that could be an evidence to explore a new candidate of cosmic ray source.

In 2011, PAMELA detector indicated that there is a breakpoint of cosmic-ray protons spectrum around 240 GV \cite{PAMELA}.
Furthermore, AMS-02 also found a drastic change of cosmic-ray proton spectrum at around 336 GV \cite{AMS-02}.
From the previous work, 5 years of \textit{Fermi} Large Area Telescope
(\textit{Fermi}-LAT) observation data has been analyzed to trace back
the characteristic of CR proton spectrum where the result imply that there is
a breaking of spectral indice around 200 GeV where the statistical significance
is around 2$\sigma$ \cite{previouswork}. In this work, 9 years of \textit{Fermi}-LAT
data would be use for finding the spectral indices of CR proton between energy
hundred MeV to a TeV range.

\section{Methodology}
\subsection{Data selection and $\gamma$-ray flux extraction}
Photon data with the latest reconstruction version (P8R2 ULTRACLEANVETO V6) of the telescope where the
observation duration takes around 9 years (starts from 7 August 2008 to 16 October 2017).
The energy range of photon is selected from 10 GeV
to 1 TeV. The upper zone of earth's limb region could be determine by
nadir angle from 68.4 to 70.0 where the coordinate figure has demonstrated in figure
\ref{gamma_production_schematic}.

\begin{figure}[h!]
    \centering
    \includegraphics[width=0.6\textwidth]{img/gamma_production_schematic}
    \caption{Schematic of $\gamma$-ray production}
    \label{gamma_production_schematic}
\end{figure}

The observed flux is defined as differential flux where the governing equation
for the calculation is represented as equation (\ref{flux_definition})
\begin{equation}
    \textbf{Flux} \equiv \frac{dN_\gamma}{dE} = \frac{\int_{\textrm{Limb region}}(\textrm{Count map}/\textrm{Exposure map})}{\Delta\Omega\Delta E }
    \label{flux_definition}
\end{equation}
Where count map is filled up with selected $\gamma$-ray and exposure map represent 
the exposure time as well as effective area of spacecraft where the angle of incident
CR has taken into account.
Procedure of computation is begin with the requirement of 25 bins of histogram of
the $\gamma$-ray flux which contain a various median of energy in each bin.
Consequently, the number of count map and exposure map will be exactly the same as
the energy bins. The calculation of exposure map is done by using log file of the
spacecraft combine with the responsiveness of the spacecraft which has to be consider
in every step time while spacecraft is online. In addition, every step time of spacecraft
require the coordinate transformation which cause a huge amount of computing process.
That is the reason why paralleling processing with Master-Slave technique is applied in this work.



\subsection{Interaction model}

Incident proton spectrum in rigidity 

In this work, we use the scattering amplitude from hadronic collision \cite{K&Omodel} that could produce a photon as a secondary product that could be detected by \textit{Fermi}-LAT.
\begin{equation}
    \frac{dN_{\gamma}}{dE_\gamma}(E_\gamma)
    \propto \sum_{E_{\textrm{inc,i}}}
    \left[\frac{E_{\text{inc,i}}}{E_{\gamma}}\Delta(E_{\textrm{inc,i}}) \right]
    % \left[ f_{pp}\textcolor{red}{\frac{dN_\textrm{H}}{dE_{\textrm{inc}}}(E_{\textrm{inc,i}})}\left\{ 1+\textcolor{olivegreen}{\frac{\sigma_{\textrm{HeN}}}{\sigma{pN}}}\left(\textcolor{red}{\frac{dN_{\textrm{H}}}{dR}}\right)^{-1} \textcolor{blue}{\frac{dN_{\text{He}}}{dR}} \frac{dR_{\textrm{He}}}{dR_{\textrm{H}}}  \right\}\right]
\end{equation}

\subsection{Optimization}




\section{Preliminary results}
The optimized parameters  for the SPL and BPL models are summarized in table~\ref{tb:bestparams}. 
The best-fit $\gamma$-ray spectra from the two models
compared to the thin-target Earth's limb measurement by the LAT are illustrated
in figure~\ref{fig:gamma-flux}, showing very similar results for both models.
Since the proton-to-$\gamma$ energy conversion factor is roughly 0.17 for broad
and smooth spectra [10], our inferred CR proton spectra are valid between 60 - 2000 GV
in rigidity
% because the lower bound is a minimum rigidity of incoming proton that could produce a 10 GeV of $\gamma$-ray
% as shown in comparison with measurements by other instruments in Figure~\ref{fig:proton-flux}.
% Note that the normalizations of our work in Figure~\ref{fig:proton-flux} are scaled
% by fitting to AMS-02 data between 100~-~2000~GV.
because the lower bound is the mean rigidity of the incoming protons that produce 10-GeV $\gamma$ rays.
Our results are shown in comparison with measurements by other instruments in figure~\ref{fig:proton-flux}.
Note that the normalizations of our work in figure~\ref{fig:proton-flux} are scaled
by fitting to AMS-02 data between 100~-~2000~GV.

% eye to approximately match PAMELA data.
% has shown in Table~\ref{tb:bestparams}.
% According to figure \ref{fig:gamma-flux}, secondary $\gamma$-ray product
% from both SPL and BPL model yield a similar product via hadronic collision
% in the atmosphere. In order to validate the indirect measurement of CR proton,
% comparing with a real observations is mandatory which has shown
% in figure \ref{fig:proton-flux}. The normalization of this work is
% fitted PAMELA data to roughly scale the incident proton spectrum.


\section{Discussion and future work}
The best-fit BPL model of CR proton from this work are consistent with direct
measurements as shown in Figure~\ref{fig:proton-flux}.
We will need to determine whether the BPL model fits the data significantly better
than the SPL model does using likelihood ratio test to quantify the level of
confidence. We also plan to determine the statistical and total uncertainties of
the fit results by performing Monte Carlo simulations. 

% From Figure~\ref{fig:gamma-flux}, a trend of incident proton model
% demonstrated that BPL has more consistency than SPL. Nevertheless,
% to determine the significance between two models require likelihood
% ratio test to evaluate the statistical level of confidence. In addition,
% statistical and total error including the instrument will be determined
% by performing Monte Carlo Simulation.

\par I would like to express my deep gratitude to 
Assist.~Prof.~Warit~Mitthumsiri and Prof.~David~Ruffolo for their patient guidance.
I would also like to thank people at the space physics laboratory at Mahidol University
for their support.
This research project is partially supported by Thailand Science Research
and Innovation (RTA6280002).




\section*{Acknowledgments}
\par I would like to express my deep gratitude to 
Assist.~Prof.~Warit~Mitthumsiri and Prof.~David~Ruffolo for their patient guidance.
I would also like to thank people at the space physics laboratory at Mahidol University
for their support.
This research project is partially supported by Thailand Science Research
and Innovation (RTA6280002).

\FloatBarrier

\section*{References}
% \bibliographystyle{abbrv}
\bibliographystyle{iopart-num}
% \bibliographystyle{unsrt}
\bibliography{iopart-num_short}

\end{document}


