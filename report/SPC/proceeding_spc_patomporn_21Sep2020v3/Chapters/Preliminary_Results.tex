The optimized parameters  for the SPL and BPL models are summarized in table~\ref{tb:bestparams}. 
The best-fit $\gamma$-ray spectra from the two models
compared to the thin-target Earth's limb measurement by the LAT are illustrated
in figure~\ref{fig:gamma-flux}, showing very similar results for both models.
Since the proton-to-$\gamma$ energy conversion factor is roughly 0.17 for broad
and smooth spectra [10], our inferred CR proton spectra are valid between 60 - 2000 GV
in rigidity
% because the lower bound is a minimum rigidity of incoming proton that could produce a 10 GeV of $\gamma$-ray
% as shown in comparison with measurements by other instruments in Figure~\ref{fig:proton-flux}.
% Note that the normalizations of our work in Figure~\ref{fig:proton-flux} are scaled
% by fitting to AMS-02 data between 100~-~2000~GV.
because the lower bound is the mean rigidity of the incoming protons that produce 10-GeV $\gamma$ rays.
Our results are shown in comparison with measurements by other instruments in figure~\ref{fig:proton-flux}.
Note that the normalizations of our work in figure~\ref{fig:proton-flux} are scaled
by fitting to AMS-02 data between 100~-~2000~GV.

% eye to approximately match PAMELA data.
% has shown in Table~\ref{tb:bestparams}.
% According to figure \ref{fig:gamma-flux}, secondary $\gamma$-ray product
% from both SPL and BPL model yield a similar product via hadronic collision
% in the atmosphere. In order to validate the indirect measurement of CR proton,
% comparing with a real observations is mandatory which has shown
% in figure \ref{fig:proton-flux}. The normalization of this work is
% fitted PAMELA data to roughly scale the incident proton spectrum.
