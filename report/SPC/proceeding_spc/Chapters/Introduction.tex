Cosmic rays (CRs) are high energy particles mainly from outer space which can sometimes
penetrate the geomagnetic field and interact with the Earth's atmosphere \cite{HESS,Pacini,Clay}.
The interactions between CRs and the air molecules produce secondary particles,
including $\gamma$ rays, mostly in the forward direction with respect to the CR velocity.
When observed from space, these CR-induced $\gamma$-ray emission of the Earth's
atmosphere appears as a bright ring along the Earth's limb due to CRs grazing
tangentially through the Earth's thin upper atmosphere and scattering photons
towards the detector. The lower atmosphere and the physical Earth create the dark
region in the center of the emission ring because they are opaque
for $\gamma$ rays (see figure 1 in \cite{Warit2009}).

There are many possible acceleration mechanisms in
space that could produce high energy particles.
The combined effects of the acceleration, propagation, and escaping from the Galaxy
result in the power-law rigidity (momentum per charge) spectrum of CRs in the
form $F\propto R^\Gamma$, where $F$ is Flux, $R$ is rigidity, and $\Gamma$ is
the spectral index. Note that for relativistic energy, the rigidity value in
the unit of GV is very close to being directly proportional to the kinetic energy
in GeV. The CR spectral index is approximately 2.7 for a very wide rigidity range,
though there are a few known changes in the index value as shown in figure~\ref{cr_knee_ankle}.
One is an abrupt softening
at $10^{15-16}$~GeV, known as the ``knee,'' \cite{Allan1962,Haungs2003} and the other one is a
hardening at $10^{18-19}$~GeV, known as the ``ankle'' \cite{ABBASI2005271}.
CRs produced by different sources or acceleration mechanisms may be characterized
by having different spectral indices. Therefore, a spectral breaking feature
could indicate a transition from a certain dominant source population of
CRs to another. For example, CRs with energy above the ankle are presumably
extragalatic. Finding spectral features will provide more clues to the origins
of CRs.


In 2011, PAMELA indicated a sudden hardening of CR proton spectrum around 240 GV \cite{PAMELA}.
Recently in 2015, AMS-02 reported the precision measurement of the CR proton spectrum
which confirmed the drastic change of the spectral index at around 336 GV \cite{AMS-02}.
As an independent cross check, in 2014 {\it Fermi} Large Area Telescope (LAT) used
about 5 years of the CR-induced Earth's $\gamma$-ray data to indirectly observe
this newly discovered spectral feature [7]. Although the best-fit broken power-law
model of CRs is consistent with the results from PAMELA and (later) AMS-02, the null
hypothesis (single power-law) is rejected at only around 1$\sigma$ \cite{previouswork}.
In this work, we improve the previous LAT analysis by using a larger data set
($\sim9$~years) and the latest version of event selection to indirectly measure the
CR proton spectral indices in the rigidity range between 60~-~2000~GV.