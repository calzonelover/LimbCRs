Cosmic-ray are high energy particle which mainly come from the outer space which can penetrate and interact with the Earth's atmosphere \cite{HESS,Pacini,Clay}.
The shark peak of gamma-ray emission from Earth's limb are mainly come from the interaction of CRs with the atmospheric molecules \cite{Warit2009}.

There are many possible phenomena of acceleration mechanism in the
space that could produce high energy particles. The characteristic of acceleration mechanism could roughly be distinguished by a spectral index in the arrival of cosmic rays spectrum in rigidity.
The breaking point of the spectrum mainly come from the overlapped region of acceleration mechanism that could be an evidence to explore a new candidate of cosmic ray source.

In 2011, PAMELA detector indicated that there is a breakpoint of cosmic-ray protons spectrum around 240 GV \cite{PAMELA}.
Furthermore, AMS-02 also found a drastic change of cosmic-ray proton spectrum at around 336 GV \cite{AMS-02}.
From the previous work, 5 years of \textit{Fermi} Large Area Telescope
(\textit{Fermi}-LAT) observation data has been analyzed to trace back
the characteristic of CR proton spectrum where the result imply that there is
a breaking of spectral indice around 200 GeV where the statistical significance
is around 2$\sigma$ \cite{previouswork}. In this work, 9 years of \textit{Fermi}-LAT
data would be use for finding the spectral indices of CR proton between energy
hundred MeV to a TeV range.