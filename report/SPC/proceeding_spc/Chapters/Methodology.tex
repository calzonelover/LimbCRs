\subsection{Data selection and $\gamma$-ray flux extraction}

We use $\sim9$ years (7 Aug 2008 - 16 Oct 2017) of the latest version
(P8R2 ULTRACLEANVETO V6) of the LAT's photon data between 10 GeV to 1 TeV.
We observe $\gamma$ rays from the Earth's thin upper atmosphere by selecting the
nadir angle ($\theta_{\rm NADIR}$) from $68.4^\circ$ to $70.0^\circ$ \cite{previouswork} as
demonstrated in Figure~\ref{gamma_production_schematic}. The incidence angle cut,
$\theta_{\rm LAT}<70^\circ$, is also applied.


% \begin{figure}[h!]
%     \centering
%     \includegraphics[width=0.5\textwidth]{img/gamma_production_schematic}
%     \caption{Schematic of $\gamma$-ray production}
%     \label{gamma_production_schematic}
% \end{figure}

% \begin{wrapfigure}{r}{0.5\textwidth}
%     \begin{center}
%         \includegraphics[width=0.5\textwidth]{img/gamma_production_schematic}
%     \end{center}
%     \caption{Schematic of $\gamma$-ray production}
%     \label{gamma_production_schematic}
% \end{wrapfigure}

\begin{figure}[h]
    \begin{minipage}{0.45\textwidth}
        \includegraphics[width=\textwidth]{img/cr_knee_ankle}
        \caption{All-particle CR spectrum taken from \cite{Swordy2001}.}
        % \caption{The all particle spectrum of cosmic rays, image taken from }
        \label{cr_knee_ankle}
    \end{minipage}\hspace{2pc}
    \begin{minipage}{0.55\textwidth}
        \includegraphics[width=\textwidth]{img/gamma_production_schematic}
        \caption{Schematic of high-energy Earth's $\gamma$-ray production}
        \label{gamma_production_schematic}
    \end{minipage} 
\end{figure}

% The observed flux is defined as differential flux where the governing equation
% for the calculation is represented as equation (\ref{flux_definition})
The observed flux for a given energy bin is calculated using
\begin{equation}
    \textbf{Flux} \equiv \frac{dN_\gamma}{dE} = \frac{\int_{\textrm{Limb region}}(\textrm{Count map}/\textrm{Exposure map})}{\Delta\Omega\Delta E }
    .\label{flux_definition}
\end{equation}

% Where count map is filled up with selected $\gamma$-ray and 
Here the count map is filled with numbers of photons, the exposure map represents 
the exposure time as well as the effective area of spacecraft 
which is a function of energy and $\theta_{\rm LAT}$, $\Delta E$ is the energy bin width,
and $\Delta\Omega$ is the solid angle of the thin-target Earth's limb region.
% Procedure of computation is begin with the requirement of 25 bins of histogram of
% the $\gamma$-ray flux which contain a various median of energy in each bin.
We perform the analysis with 25 bins of energy, equally spaced in logarithmic scale.
% Consequently, the number of count map and exposure map will be exactly the same as
% the energy bins. The calculation of exposure map is done by using log file of the
% spacecraft combine with the responsiveness of the spacecraft which has to be consider
% in every step time while spacecraft is online. 
For a given energy bin, the exposure map is calculated using the spacecraft's position
and orientation recorded in 30-second time steps, each of which involves a complex
coordinate transformation to create a map in the zenith-azimuth system.
Such computationally intensive task requires parallel
processing with Master-Slave technique that we have developed.
% which cause a huge amount of computing process.
% That is the reason why paralleling processing with Master-Slave technique is
% applied in this work.


\subsection{Interaction model}
In this work, we test 2 models of CR protons: single-power law (SPL) model
containing one spectral index, and broken-power law (SPL) model containing two
spectral indices with a break energy.

% The model for a scattering amplitude from hadronic collision \cite{K&Omodel}
% that could produce a photon as a secondary product which could be
% detected by \textit{Fermi}-LAT as equation \ref{eq:interaction_model}.
According to \cite{K&Omodel}, the secondary photon spectrum from proton-proton
collisions could be summarized by

\begin{equation}
    \frac{dN_\gamma}{dE_\gamma}\propto \int^{E_{\text{max}}}_{E_\gamma} dE'\frac{dN_p}{dE'} \frac{d\sigma^{pp\rightarrow\gamma}(E',E_\gamma)}{dE_\gamma}
    ,\label{eq:interaction_model}
\end{equation}

where here $dN_\gamma/dE_\gamma$ is the measured Earth's limb $\gamma$-ray spectrum,
$dN_p/dE'$ is the CR proton model, and $\sigma^{pp\rightarrow\gamma}$ is the interaction
cross section.
We take into account the contribution from CR He particles to the production of secondary
photons by using the cross section ratio ($\sigma_{\rm HeN}/\sigma_{p\rm N}$) from
\cite{WAtwater} and the He spectrum measurement by \cite{AMS-02Helium}. This modifies
Eq.~(\ref{eq:interaction_model}) to
% For the real use case, the interaction of an alphaparticle with the air
% has a significant contribution to the secondary photon.
% A modification of He-air interaction could be
% applied by using a fraction of cross-section from a given atomic number
% \cite{WAtwater}. The helium spectrum in rigidity is taken from the real
% measurement \cite{AMS-02Helium}. Then the input of the modified model
% is left only a proton spectrum.

\begin{equation}
    \frac{dN_{\gamma}}{dE_\gamma}(E_\gamma) \propto
    \sum_{E'_i}\left[\frac{E'_i}{E_{\gamma}}\Delta(\ln E'_i) \right]
    \left[ 
        f_{pp}\frac{dN_p}{dE'_i}
        \left\{
            1+\frac{\sigma_{\text{HeN}}}{\sigma{p\rm N}}\left(\frac{dN_p}{dR}\right)^{-1} \frac{dN_{\text{He}}}{dR} \frac{dR_{\text{He}}}{dR_p} 
        \right\}
    \right]
    ,\label{eq:derived_model}
\end{equation}

where $f_{pp} \equiv E_\gamma(d\sigma^{ij\rightarrow\gamma}/dE_\gamma)$
is the interaction cross section table in \cite{K&Omodel}.

\subsection{Optimization}

We use the SPL and BPL models for $dN_p/dE'$ in Eq.~(\ref{eq:derived_model}) and vary their parameters
(normalization, spectral indices, break energy) so that the resulting $dN_\gamma/dE_\gamma$
from the model fits to the measured Earth's limb $\gamma$-ray spectrum with maximum
likelihood. We employ the particle swarm optimization (PSO) \cite{pso_optimize} as our fitting algorithm
because PSO is efficient at avoiding local maxima and reaching the global maximum in
this multi-parameter problem.
% Optimizing a problem with multiple parameters might cause a local minimum
% which will cause an early stopping of the optimization proces before 
% reaching to the global minimum. In this work, a simple gradient optimization
% with a set of different initial values yield a various output which implicitly
% imply that there are local minimum exists in this problem. 
% To get rid of the local minimum, particle swarm optimization (PSO) is applied
% to find the best fit parameters \cite{pso_optimize}.



