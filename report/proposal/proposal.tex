\documentclass[12pt, a4paper]{article}
\setlength{\oddsidemargin}{0.5cm}
\setlength{\evensidemargin}{0.5cm}
\setlength{\topmargin}{-1.6cm}
\setlength{\leftmargin}{0.5cm}
\setlength{\rightmargin}{0.5cm}
\setlength{\textheight}{24.00cm} 
\setlength{\textwidth}{15.00cm}
\parindent 0pt
\parskip 5pt
\pagestyle{plain}
\usepackage[table,xcdraw]{xcolor}
\usepackage{tikz}
\usepackage{amsmath}
\def\checkmark{\tikz\fill[scale=0.4](0,.35) -- (.25,0) -- (1,.7) -- (.25,.15) -- cycle;}

\usepackage[utf8]{inputenc}
\usepackage[english]{babel}
 
\setlength{\parindent}{4em}
\setlength{\parskip}{1em}
%\renewcommand{\baselinestretch}{2.0}
\usepackage{indentfirst}

\usepackage{bm}
\newcommand{\uveci}{{\bm{\hat{\textnormal{\bfseries\i}}}}}
\newcommand{\uvecj}{{\bm{\hat{\textnormal{\bfseries\j}}}}}
\DeclareRobustCommand{\uvec}[1]{{%
  \ifcsname uvec#1\endcsname
     \csname uvec#1\endcsname
   \else
    \bm{\hat{\mathbf{#1}}}%
   \fi
}}




\title{THESIS PROPOSAL}
\author{}
\date{}

\newcommand{\namelistlabel}[1]{\mbox{#1}\hfil}
\newenvironment{namelist}[1]{%1
\begin{list}{}
    {
        \let\makelabel\namelistlabel
        \settowidth{\labelwidth}{#1}
        \setlength{\leftmargin}{1.1\labelwidth}
    }
  }{%1
\end{list}}

\begin{document}
\maketitle

\begin{namelist}{xxxxxxxxxxxx}
\item[{\bf Title:}]
	Indirect measurement of cosmic-ray proton spectrum using gamma-ray data from Fermi Large Area Telescope
\item[{\bf Student:}]
	Patomporn Payoungkhamdee 6138171 SCPY/M
\item[{\bf Supervisor:}]
	Assistance Professor Warit Mitthumsiri
\item[{\bf Degree:}]
	Master's degree
\item[{\bf Field of study:}]
	Physics
\item[{\bf Faculty of Science,  Mahidol University }]
\end{namelist}

\section{Introduction}


\section{Methodology and Scope}
\subsection*{Methodology}
\subsubsection*{ An introduction to Particle In Cell Model}
%-Scope:J,E,v ions and electrons, dens ions and electron, B

\section{Research planning}

\begin{thebibliography}{9}
%Abraham C.-L. Chian and Pablo R.Muñoz. “Detection of current sheets and magnetic reconnections at %the turbulent leading edge of an interplanetary coronal mass ejection.” Astrophysics Journal %Letters 733: L34 (5pp), 2011 June 1.
%        MalakitK.2012.PhD.Thesis, “A symmetric magnetic reconnection: A particle-in-cell study”

\bibitem{Chian} Abraham C.-L. Chian and Pablo R.Mu$\tilde{n}$oz. {\em “Detection of current sheets and magnetic reconnections at the turbulent leading edge of an interplanetary coronal mass ejection." }\/ Astrophysics Journal Letters 733: L34 (5pp), 2011 June 1.
\bibitem{Malakit} Malakit, K. {\em “Asymmetric magnetic reconnection: A particle-in-cell study.” }\/ PhD.Thesis, 2012.
\bibitem{SpaceWeather} Space Weather, {\em “ What's up in space$--$21 Jan 2005.” }\/\\ http://spaceweather.com/archive.php?view=1\&day=12\&month=09\&year=2017
\bibitem{NAT} Gopalswamy, N. {\em “ Properties of Interplanetary Coronal Mass Ejection."}\/ Space Science Review (2006) 124: 145. https://doi.org/10.1007/s11214-006-9102-1
%Gopalswamy, N. Space Sci Rev (2006) 124: 145. https://doi.org/10.1007/s11214-006-9102-1

%or Logical, {\em Notices of the Amer. Maths. Soc.},\/ Vol. 34,
%1987, pp. 621-624.
\end{thebibliography}


\end{document}

