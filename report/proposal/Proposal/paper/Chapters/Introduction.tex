\par Cosmic-ray research has been pioneered by Theodor Wulf and Victor Hess who took electrometer measured cosmic rays from the ground to a higher altitude and later experiment has confirmed that there are a cosmic rays from outer space which can penetrate and interact with the Earth's atmosphere \cite{HESS,Pacini,Clay}.

\par There are many possible phenomena of acceleration mechanism in the
space that could produce high energy particles. Consequently, characteristic of acceleration mechanism could roughly be distinguished by a spectral index in the arrival of cosmic rays spectrum in rigidity.
The breaking point of the spectrum mainly come from the overlapped region of acceleration mechanism that could be an evidence to explore a new candidate of cosmic ray source.

In 2011, PAMELA detector indicated that there is a breakpoint of cosmic-ray protons spectrum around 240 GV \cite{PAMELA}.
Furthermore, AMS-02 also found a drastic change of cosmic-ray proton spectrum at around 336 GV \cite{AMS-02}.

\par In this work, the indirect measurement of cosmic ray protons will be performed by using gamma-ray data from \textit{Fermi} Large Area Telescope (\textit{Fermi}-LAT).