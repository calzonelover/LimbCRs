\documentclass[a4paper, 12pt]{article}
\usepackage[margin=25mm]{geometry}
\usepackage{amsmath}
\usepackage{amsfonts}
\usepackage{amssymb}
\usepackage{graphicx}
\pagenumbering{gobble}
\usepackage{verbatim}
\immediate\write18{texcount -tex -sum  \jobname.tex > \jobname.wordcount.tex}

% Keywords command
\providecommand{\keywords}[1]
{
  \small	
  \textbf{\textit{Keywords---}} #1
}

\title{Preliminary measurement of cosmic-ray proton spectrum using $\gamma$-ray data from {\it Fermi} Large Area Telescope}
\author{
  Patomporn Payoungkhamdee$^{\dagger}$,\\
  % under supervision of\\
  Asst. Prof. Warit Mitthumsiri$^{\dagger}$, Prof. David John Ruffolo$^{\dagger}$\\
  \small $^{\dagger}$Department of Physics, Faculty of Science, Mahidol University
  % \small $^{2}$University B \\
}
\date{} % Comment this line to show today's date

\begin{document}
\maketitle

\begin{abstract}
\normalsize
% Cosmic rays (CRs) are high-energy particles propagating in space. It is mainly constituted by protons and the rigidity spectrum is well described by a power law. Recent
% measurements by PAMELA and AMS-02 indicate an abrupt change of the CR proton
% spectral index at about 336 GV. When protons interact with the Earth’s upper atmosphere, $\gamma$-rays can be produced and detected by space-based detectors. The
% Earth Limb $\gamma$-ray data was collected by the {\it Fermi} Large Area Telescope (LAT) along with
% proton-air interaction models to determine the CR proton spectral indices that best fit
% the $\gamma$-ray data.

Cosmic rays (CRs) are high-energy particles, mostly protons, propagating in space. The rigidity (momentum per charge) spectrum of CRs is well described by a power law for which the spectral index is approximately 2.8 around 30 - 1000 GV. Recent measurements by PAMELA and AMS-02 indicate an abrupt change of the CR proton spectral index at about 340 GV. When CRs interact with the Earth's upper atmosphere, $\gamma$ rays can be produced and detected by space-based detectors. Here we use the Earth's $\gamma$-ray data collected by the {\it Fermi} Large Area Telescope along with a proton-air interaction model to indirectly determine the CR proton spectral index and compare against observations by other instruments.
This research project is partially supported by Thailand Science Research and Innovation (RTA6280002).

\end{abstract}
\hspace{10pt}

%TC:ignore
% Keyword:
\keywords{ Cosmic rays, gamma-rays, {\it Fermi}-LAT}
% \begin{thebibliography}{00}
% \bibitem{} M. Ackermann et al., (Fermi LAT Collaboration), Phys. Rev. Lett. 112, 151103 (2014)
% \bibitem{} M. Aguilar et al. (AMS Collaboration), Phys. Rev. Lett. 114, 171103 (2015)
% \bibitem{} J. Alcaraz et al., Physics Letters B 490, 27 (2000)
% \bibitem{} S. Haino et al., Physics Letters B 594, 35 (2004)
% \end{thebibliography}
%TC:endignore


% Word count
\verbatiminput{\jobname.wordcount.tex}

\end{document}