\chapter{Power law in energy}
\label{appendix:pw_energy}

Equation \ref{eq:spl} and Equation \ref{eq:bpl}
express the spectral power law in rigidity.
We can derive them
% The relation of the 
% power-law spectrum in rigidity could be derived
into the 
energy spectrum as in Equation \ref{eq:spl_energy} and 
Equation \ref{eq:bpl_energy} respectively. 

\textbf{Single power law (SPL)}

The energy spectrum in energy is derived as

\begin{equation}
    \frac{dN}{dE} = N_0[E_k(E_k+2m_p)]^{-\gamma/2} \left(\frac{E_k+m_p}{\sqrt{E_k(E_k+2m_p)}}\right)
    \label{eq:spl_energy}
\end{equation}

where $m_p$ is the mass of proton and $E_k$ is the kinetic 
energy and $N_0$ is defined as the normalization factor.


\textbf{Broken power law (BPL)}

\begin{equation}
\frac{dN}{dE}=
  \begin{cases}
    N_0[E_k(E_k+2m_p)]^{-\gamma_1/2} \left(\frac{E_k+m_p}{\sqrt{E_k(E_k+2m_p)}}\right)\ :\ E < E_{\text{Break}}\\
    N_0[E_b(E_b+2m_p)]^{(\gamma_2-\gamma_1)/2}[E_k(E_k+2m_p)]^{-\gamma_2/2} \left(\frac{E_k+m_p}{\sqrt{E_k(E_k+2m_p)}}\right)\\ :\ E \ge E_{\text{Break}}
  \end{cases}
  \label{eq:bpl_energy}
\end{equation}

where the symbols are defined in the same way as Equation \ref{eq:spl_energy}.
