\chapter{Conclusion}


% Overview of what we have study
In this study, the incident CR proton spectrum was tracing by 
using the limb $\gamma$-ray. The measured $\gamma$-ray spectrum
is constructing by taking the exposure of the LAT and
the effective area into account where on the Earth-centered 
coordinates. The exposure calculation is calculated from the 
relation of the angle between the LAT's boresight and nadir angle 
because given nadir angle and the spacecraft orientation affects the 
performance of the measurement. After that, an indirect measurement 
of the proton spectrum was inversely computed via the
$pp\rightarrow\gamma$ interaction model with a heuristic optimization
technique.


% fermi-lat mission -> gamma induction -> this study
Since the \textit{Fermi}-LAT was launched in June 2008.
The data has been collected from 7 August 2008 to 16 October 2017.
The collection from 9 years of observations was recorded for both 
$\gamma$-rays data and the spacecraft orientation logging.
The LAT mostly observing the sky and looking for a flare of $\gamma$-rays.
However, there is a moment that LAT FoV can see the Earth or even stare at the Earth for some reason.
Filtering photon from the Earth's limb region was done by selecting 
an incoming photon in the direction between 68.4\textdegree -
70\textdegree nadir angle with the cleanest event class.


% Analysis and the calculation


% found that xx with condidence level xx
% describe and acompare to previous study and with other experiment
To sum up, a statistical significance from the analysis yields 
1.38$\sigma$. Meaning that there is a confidence level of breaking
spectrum in CR proton at 92\%. This level of significance is still 
now strong for confirmation. Nevertheless, the previous 
study with a few data collection periods gives 1.0$\sigma$ significant
level. It implicitly tells that the larger data collection could 
increase the statistics and put the weight on the previous study.
Surprisingly, the direct CR proton measurement from AMS-02 identify 
a breaking spectrum at 340 GV which is very close to our work.
There is also one plausible assumption where the indirect measurement
could not yield a strong significant even we have a huge amount of data.

