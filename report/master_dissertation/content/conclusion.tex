\chapter{Conclusion}


% Overview of what we have study
In this study,
we indirectly measure the CR proton spectrum between 60 GeV -- 2 TeV
using the Earth's $\gamma$-ray spectrum with the LAT data.
We construct the Earth's $\gamma$-ray spectrum by analyzing
the photon count and the LAT exposure above 10 GeV in
the local zenith coordinates from 9 years of data (Aug 2008 to Oct 2017).
We use photons of the cleanest event class (P8UltracleanVeto)
which are produced by the interactions of CRs and
the Earth's upper atmosphere in the thin-target regime
($68.4^\circ<\theta_{\rm NADIR}<70.0^\circ$).
% the incident CR proton spectrum was tracing by 
% using the limb $\gamma$-ray. The measured $\gamma$-ray spectrum
% is constructing by taking the exposure of the LAT and
% the effective area into account where on the Earth-centered 
% coordinates. The exposure calculation is calculated from the 
% relation of the angle between the LAT's boresight and nadir angle 
% because given nadir angle and the spacecraft orientation affects the 
% performance of the measurement.
An indirect measurement 
of the proton spectrum is inversely computed via the
$pp\rightarrow\gamma$ interaction model with a heuristic optimization
technique.


% fermi-lat mission -> gamma induction -> this study
% Since the \textit{Fermi}-LAT was launched in June 2008.
% The data that has been used in this analysis starting
% from 7 August 2008 to 16 October 2017.
% The collection from 9 years of observations was recorded for both 
% $\gamma$-rays data and the spacecraft orientation logging.
% The LAT mostly observing the sky and looking for a flare of $\gamma$-rays.
% However, there is a moment that LAT FoV can see the Earth or even stare at the Earth for some reason.
% Filtering photons from the Earth's limb region was done by selecting 
% an incoming photon in the direction between 68.4\textdegree -
% 70\textdegree nadir angle with the cleanest event class.


% Analysis and the calculation


% found that xx with condidence level xx
% describe and acompare to previous study and with other experiment


Our best-fit single power law (SPL) model for CR proton yields
the spectral index value of $2.70\pm 0.08$, while the
best-fit broken power law (BPL) model yields a hardening of
CR proton from the spectral index of $2.86\pm 0.14$ to $2.63\pm 0.13$
at $333\pm 10$ GeV. The BPL model fits the data better than
the SPL model does at the statistical significance of 1.38$\sigma$,
corresponding 92\% confidence level which according to
standard convention is not strong enough to conclusively declare
the spectral break from this analysis alone.
Nevertheless, we confirm the best-fit models and improve
the significance level from $1.0\sigma$ in the previous study by the LAT.
Although precision measurement by AMS-02 and other
direct measurements have settled the existence of
this CR proton spectral break at approximately 340 GV,
having independent supports from different experiments
such as this work is valuable. Spectral features will help us
understand the origins and propagation of Galactic CRs in the future.
% To sum up, a statistical significance from the analysis yields 
% 1.38$\sigma$. Meaning that there is a confidence level of breaking
% spectrum in CR proton at 92\%. This level of significance is still 
% now strong for confirmation. Nevertheless, the previous 
% study with a few data collection periods gives 1.0$\sigma$ significant
% level. It implicitly tells that the larger data collection could 
% increase the statistics and put the weight on the previous study.
% Surprisingly, the direct CR proton measurement from AMS-02 identifies 
% a breaking spectrum at 340 GV which is very close to our work.
% There is also one plausible assumption where the indirect measurement
% could not yield a strong significant even have a huge amount of data.

