\chapter{Introduction}


%what why where when how
\section{Overview}
% - Astrophysical phonomenon and nature with CR
Space is full of fascinating phenomena and the questions,
from why stars are bright to
more advanced questions such as the nature of dark matter.
Human curiosity has brought us so far that now we
can observe the sky with more sophisticated techniques.
However, the research to gather new knowledge by studying
the space is endless. The answer to one question sometimes
generates another mystery. Research in physical science
also helps to create new technology because of the need
to overcome challenging limitations of the instruments or techniques.

% Astrophysics -> CR study
The frontier of astrophysical research is continually expanding over time
because the exploration of one thing does open the new door 
to another dark room which has been waiting for human to shine a light to explore.
There are various branches in astrophysics from theoretical
foundation, simulation and experimental physics
which all complement of them trying to push the frontier of human knowledge.
To study high energy particle accelerators in the universe, the 
possibility of direct probing of multiple Galactic sources which
produce high-energy particles, or cosmic rays (CRs), is nearly
impossible in terms of current technology and resources required.
Nevertheless, the technology of observing the 
particles arriving the Earth is more plausible for scientists.


% .... describe more! 
CRs can be observed with two types of detectors: ground-based
and spaced-based.  Analyzing and studying CR data allows
us to interpret the properties of their sources and cosmic environment.

The spectrum of CRs follows a power law with different spectral
indices depending on the rigidity (momentum per charge) range of
particles. There are multiple types of CR sources in space
including unknown sources.
% There are multiple types of CR sources in space including unknown sources.
Consequently, changing of 
the spectral index from one rigidity to another rigidity will find
the discontinuity if there is the translation from one source type 
to another source from the superposition of multiple spectrums.


% - Earth's limb gamma ray and previous study
\textit{Fermi}-LAT has been launched to orbit around the
Earth and monitor the $\gamma$-ray sky.
Interestingly, the brightest $\gamma$-ray source in the sky for
the LAT is the Earth's limb due to its proximity. At above ~1 GeV,
the Earth's $\gamma$-ray emission from CRs interacting with the
upper atmosphere appears as a bright ring \citep{FermiEarth09}.
% It found that the ring of brightness around the Earth's limb where 
% the major factor that causes this phenomenon is the interaction of 
% incoming CRs with the Earth's upper atmosphere as analyzed in \cite{FermiEarth09}.
% Then the spectrum of $\gamma$-rays that was induced by the incoming 
% CRs are highly related to the spectrum of CRs.



Before 2010, there were some hints of the abrupt change in the CR
spectral index at~300 GV in rigidity by some experiments
\citep{adriani2011pamela,cream2004,atic2002,bess_experiment},
% (xxcite PAMELA, CREAM, ATIC, BESS)
though the conclusion was
from the combination of data from different experiments which
is prone to the systematic uncertainty. In 2014, \textit{Fermi}-LAT
attempted to measure this spectral feature indirectly using
the Earth's $\gamma$-ray emission data from 5 years of
observations \citep{FermiEarth14}. The inferred spectral
indices are consistent with other experiment, showing a spectral
break at around 300 GV with $\approx1\sigma$ significance level,
which is not high enough to make a definitive conclusion.

% The first indirect measurement was conducted with 5 years of
% observations and indicate that there is a breaking of spectral
% index around 302 GV with a significant level 1 $\sigma$.
% The significant level at this stage is not so strong to conclude the 
% study. The reasons probably came from the nature of the CRs if there 
% is no discontinuity in the incident CRs spectrum, the indirect could 
% measurement distort the information so that the sign could be reducing
% or the exposure time during the observations is still not enough.
% To confirm that we did our best on the data
% collection side, performing the analysis with more data could also 
% put us out of doubt for the last clue.


\section{Objectives}
The objectives of this study are to 
\begin{itemize}
    \item To indirectly measure the CR proton spectrum between approximately 60 -- 1000 GV in rigidity.
    \item To build on the results from the previous study with more dataset
    \item To improve the optimization technique by using the heuristical methodology
    \item To reduce the calculation time by inventing a new
    parallel code in low level from scratch
\end{itemize}


\section{Outline of thesis}
The dissertation provides various information from the
overview introductory context to the technical detail employed
in this study as well as the results and interpretations. It is 
structured as follows.

Chapter I provides the overview and objectives of this work.

Chapter II is the background knowledge relating to this 
study. This chapter also provides brief history of cosmic ray 
research which contains impactful experiments and important
findings which have advanced the field.
Some theoretical detail will be provided along with the historical
discoveries. Subchapters describe in more detail about high-energy
astroparticle physics.

Chapter III consists of multiple literature reviews involving
the study to clarify the theoretical idea as well as for filling the
concepts which are important for our better understanding of the next chapter.


% % Literature review
% Chapter IV would cover the article reviewing.
% The content is staring from the pioneering article of the 
% field and the evolution that inspire this work.

% V Methodology
Chapter IV consists of datasets selection, flux calculation
, computation optimization and interpretation. The following chapter which is chapter V will be the discussion from the analyzed results.
The last chapter 

The last chapter (Chapter V) is the final summarization from this study.
It composed of the final numerical results and the statistical significance.
