\chapter{Introduction}


%what why where when how
\section{Overview}
% - Astrophysical phonomenon and nature with CR
The space is full of questionable amazing phonomenon starting
from why we have day and night, why the star bright to 
the advance question about existence of the dark matter.
This influence to the mankind to start asking how it happen.
Human curiosity bring us so far that now we
could observe the sky with high resolution technique.
However, the research to gather new knowledge by studying
the space is endless and the answering one thing usually
bring another mystery. Study of physical science also
drag the technology for further step because the limitation
of the instrument or technique to explore the nature is 
another challenging task.

% Astrophysics -> CR study
Astrophysical research boundary is expanding through time pass by
because the exploration of one thing does open the new door 
with the darky room waiting for human to shine a light to explore.
There are various branch of astrophysical science from theoretical
foundation, simulation and experimental physics where they are
compliment each other for pushing the frontier of the human knowledge.
To study high energy particle accelerators in the universe, the 
possibility of probing the source is nearly impossible in terms of 
current technology and resource that required to reach multiple 
galactic sources that could produce high energy particle called
cosmic-rays (CRs). Nevertheless, the technology of observing the 
particles that arriving the Earth is more plausible for the scientist.


% .... describe more! 
The way to observe cosmic-ray (CR) particles has divided 
into two ways. One way is measuring the incoming charge particles on
ground with a ground-base detector. Alternative way is 
gathering CRs on the space or basically orbiting spacecraft.
Performing analysis on CRs data allow us to interpret the
physical properties of CRs particle for both quantatively
and qualitatively by taking simulation and experimental
results to compare.

Typically, the spectrum of the CRs would follow the power law with a
specific spectral index depending on the rigidity of the charged particles
where it was accelerated by a specific source. It is obvious that there
will be multiple sources of the CRs in the space including unknown sources.
The characteristic of spectrum would be indicated by the source from 
theoretical simulation and derivations. Consequencely, changing of 
the spectral index from one rigidity to another rigidity will find
the discontinuity if there are the translation from one source type 
to another source from the superposition of multiple spectrums.


% - Earth's limb gamma ray and previous study
\textit{Fermi}-LAT has been launched into the sky and orbiting around the
Earth and looking around the space in $\gamma$-rays regions.
It found that the ring of brightness around the Earth's limb where 
the major factor that cause this phenomenon is the intereaction of 
incoming CRs with the Earth's upper atmosphere as analyzed in \cite{FermiEarth09}.
Then the spectrum of $\gamma$-rays that was induced by the incoming 
CRs highly related on the spectrum of CRs.


The first indirect measurement was conducted with 5 years of
observations and indicate that there is breaking of spectral
index around 302 GV with a significant level 1 $\sigma$.
The significant level at this stage is not so strong to conclude the 
study. The reasons probably came from the nature of the CRs if there 
is no discontinuity in the incident CRs spectrum, the indirect could 
measurement distort the information so that the significant could be reducing
or the exposure time during the observations is still not enough.
In order to confirm that we did our best on the data
collection side, performing the analysis with more data could also 
put us out doubt for the last clue.


\section{Objectives}
The objectives of this study are to 
\begin{itemize}
    \item To indirectly measure the cosmic ray proton spectrum in rigidity range gigaelectronvolt (GV)
    \item To put the weight on the previous study with more dataset
    \item To improve the optimizaiton technique by using heuristical methodology
    \item To reduce the calculation time by inventing a whole new
    parallel code in low level from scratch
\end{itemize}


\section{Outline of Thesis}
The dissertation would provide the various information from the 
overview introductory context to the technical detail that employ 
in this study as well as the result and interpretation. It is 
structured as follows.

Chapter I will introduce the reader about the overview of the long 
track from the historical analogy and zooming into the specific 
branch of research to get the reader to see where we are and what 
we are doing to fullfill the frontier of the research.

Chapter II is the background knowledge that will be used in this 
study. This chapter also have a brief of history in cosmic ray 
research community which contains an important finding and the 
impactful experiment that bring us to this far in the research field.
Some theoretical detail will be provided on par with the historical
discovery but the subchapter of the specific topic will describe more
detail in depth of astroparticle physics that involving the high 
energy physics. Not only the concept of physical process but this 
chapter also have the apparatus informations where the majority of the
content covering the detector part in the spacecraft to demonstrate
how the apparatus gather the $\gamma$-rays data.

Chapter III is mainly consist of multiple literature reviews involving
the study to clarify the theoretical idea as well as for filling the
fundamental concept that takes reader to understand the next chapter 
in detail. 


% Literature review
Chapter IV would cover the article reviewing.
The content is staring from pioneering article of the 
field and the evolution that inspire this work.

% V Methodology
Chapter V consists of datasets selection, flux calculation
, problem optimizaiton and interpretation.

