\chapter{Introduction}


%what why where when how
\section{Overview}
% - Astrophysical phonomenon and nature with CR
Space is full of fascinating phenomena, why stars are bright to
more advanced questions such as whether dark matter exists.
Human curiosity has brought us so far that now we
can observe the sky with more sophisticated techniques.
However, the research to gather new knowledge by studying
the space is endless. The answer to one question sometimes
generate another mystery. Research in physical science
also helps creating new technology because of the need
to overcome challenging limitations of the instruments or techniques.

% Astrophysics -> CR study
The frontier of astrophysical research is continually expanding over time
because the exploration of one thing does open the new door 
to another dark room which has been waiting for human to shine a light to explore.
There are various branches in astrophysics from theoretical
foundation, simulation and experimental physics which all
compliment each other for pushing the frontier of the human knowledge.
To study high energy particle accelerators in the universe, the 
possibility of direct probing of multiple Galactic sources which
produce high-energy particles, or cosmic rays (CRs), is nearly
impossible in terms of current technology and resources required.
Nevertheless, the technology of observing the 
particles arriving the Earth is more plausible for scientists.


% .... describe more! 
CRs can be observed with two types of detectors: ground-based
and spaced-based.  Analyzing and studying CR data allows
us to interpret properties of their sources and cosmic environment.

The spectrum of CRs follows a power law with different spectral
indices depending on the rigidity (momentum per charge) range of
particles. There are multiple types of CR sources in space
including unknown sources.
There are multiple types of CR sources in space including unknown sources.
Consequently, changing of 
the spectral index from one rigidity to another rigidity will find
the discontinuity if there is the translation from one source type 
to another source from the superposition of multiple spectrums.


% - Earth's limb gamma ray and previous study
\textit{Fermi}-LAT has been launched into the sky and orbiting around the
Earth and looking around the space in $\gamma$-rays regions.
It found that the ring of brightness around the Earth's limb where 
the major factor that causes this phenomenon is the interaction of 
incoming CRs with the Earth's upper atmosphere as analyzed in \cite{FermiEarth09}.
Then the spectrum of $\gamma$-rays that was induced by the incoming 
CRs are highly related to the spectrum of CRs.


The first indirect measurement was conducted with 5 years of
observations and indicate that there is a breaking of spectral
index around 302 GV with a significant level 1 $\sigma$.
The significant level at this stage is not so strong to conclude the 
study. The reasons probably came from the nature of the CRs if there 
is no discontinuity in the incident CRs spectrum, the indirect could 
measurement distort the information so that the sign could be reducing
or the exposure time during the observations is still not enough.
To confirm that we did our best on the data
collection side, performing the analysis with more data could also 
put us out of doubt for the last clue.


\section{Objectives}
The objectives of this study are to 
\begin{itemize}
    \item To indirectly measure the cosmic ray proton spectrum in rigidity range gigaelectronvolt (GV)
    \item To put the weight on the previous study with more dataset
    \item To improve the optimizaiton technique by using heuristical methodology
    \item To reduce the calculation time by inventing a whole new
    parallel code in low level from scratch
\end{itemize}


\section{Outline of Thesis}
The dissertation would provide the various information from the
overview introductory context to the technical detail employ
in this study as well as the result and interpretation. It is 
structured as follows.

Chapter I will introduce the reader to the overview of the long 
track from the historical analogy and zooming into the specific 
branch of research to get the reader to see where we are and what 
we are doing to fulfill the frontier of the research.

Chapter II is the background knowledge that will be used in this 
study. This chapter also has a brief of history in cosmic ray 
research community which contains an important finding and the 
impactful experiment that brings us to this far in the research field.
Some theoretical detail will be provided on par with the historical
discovery but the subchapter of the specific topic will describe more
detail in the depth of astroparticle physics that involving the high 
energy physics. Not only the concept of physical process but this 
chapter also has the apparatus information where the majority of the
content covering the detector part in the spacecraft to demonstrate
how the apparatus gather the $\gamma$-rays data.

Chapter III is mainly consist of multiple literature reviews involving
the study to clarify the theoretical idea as well as for filling the
fundamental concept that takes the reader to understand the next chapter 
in detail. 


% Literature review
Chapter IV would cover the article reviewing.
The content is staring from the pioneering article of the 
field and the evolution that inspire this work.

% V Methodology
Chapter V consists of datasets selection, flux calculation
, problem optimization and interpretation.

