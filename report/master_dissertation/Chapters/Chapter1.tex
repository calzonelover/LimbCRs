\chapter{Introduction}
\lhead{Introduction}

% describe cosmic ray study since past to now just brief
\par Since the study of \acf{CR} has been led by Theodor Wulf who taked electrometer measured \acf{CR} from the ground to higher altitude and much more experiment has confirmed that there is a cosmic ray from outer space which can penetrate to the Earth's surface \cite{HESS,Pacini,Clay}.

% Consequently
\par Consequently, there are many possible phenomena of acceleration mechanism in the space that could produced high energy particles which would be a \acs{CR}. In order to study the characteristic of acceleration mechanism, we could take consider \acs{CR} spectrum which has a unique energy break point of superposition between different phenomena.

In 2011, \acs{Pamela} detecter indicate that there is a break point of proton cosmic ray spectrum around 240 GV \cite{PAMELA}. Furthermore, AMS-02 also found a drastic change of proton CR spectrum at around 336 GV. \cite{AMS-02}
% Early study, and what we gonna do
\par \acs{CR} are mainly compose with proton and 10$\%$ by Helium. Generally, \acf{gmr} could be produced by \acs{CR} collide with Earth's upper atmosphere. In this study, we perform an indirect measurement of proton spectrum from \acs{gmr} data which collected by \acf{Fermi-LAT} to find best fit incident proton spectrum that best fit to \acs{gmr} spectrum.
