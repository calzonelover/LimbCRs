%%%%%%%%%%%%%%%%%%%%%%%%%%%%%%%%%%%%%%%%%
% Beamer Presentation
% LaTeX Template
% Version 1.0 (10/11/12)
%
% This template has been downloaded from:
% http://www.LaTeXTemplates.com
%
% License:
% CC BY-NC-SA 3.0 (http://creativecommons.org/licenses/by-nc-sa/3.0/)
%
%%%%%%%%%%%%%%%%%%%%%%%%%%%%%%%%%%%%%%%%%

%----------------------------------------------------------------------------------------
%	PACKAGES AND THEMES
%----------------------------------------------------------------------------------------
\documentclass{beamer}

\mode<presentation> {

% The Beamer class comes with a number of default slide themes
% which change the colors and layouts of slides. Below this is a list
% of all the themes, uncomment each in turn to see what they look like.

% \usetheme{default}
%\usetheme{AnnArbor}
%\usetheme{Antibes}
%\usetheme{Bergen}
%\usetheme{Berkeley}
%\usetheme{Berlin}
%\usetheme{Boadilla}
% \usetheme{CambridgeUS}
% \usetheme{Copenhagen}
% \usetheme{Darmstadt}
%\usetheme{Dresden}
%\usetheme{Frankfurt}
%\usetheme{Goettingen}
%\usetheme{Hannover}
%\usetheme{Ilmenau}
%\usetheme{JuanLesPins}
%\usetheme{Luebeck}
\usetheme{Madrid}
%\usetheme{Malmoe}
%\usetheme{Marburg}
%\usetheme{Montpellier}
%\usetheme{PaloAlto}
%\usetheme{Pittsburgh}
%\usetheme{Rochester}
%\usetheme{Singapore}
%\usetheme{Szeged}
%\usetheme{Warsaw}

% As well as themes, the Beamer class has a number of color themes
% for any slide theme. Uncomment each of these in turn to see how it
% changes the colors of your current slide theme.

%\usecolortheme{albatross}
%\usecolortheme{beaver}
%\usecolortheme{beetle}
%\usecolortheme{crane}
%\usecolortheme{dolphin}
%\usecolortheme{dove}
%\usecolortheme{fly}
%\usecolortheme{lily}
%\usecolortheme{orchid}
%\usecolortheme{rose}
%\usecolortheme{seagull}
%\usecolortheme{seahorse}
%\usecolortheme{whale}
%\usecolortheme{wolverine}

%\setbeamertemplate{footline} % To remove the footer line in all slides uncomment this line
%\setbeamertemplate{footline}[page number] % To replace the footer line in all slides with a simple slide count uncomment this line

%\setbeamertemplate{navigation symbols}{} % To remove the navigation symbols from the bottom of all slides uncomment this line
}

% tikz and flowchart
\usepackage{tikz}
\usetikzlibrary{shapes.geometric, arrows}

\usepackage{transparent}
\usepackage{overpic}
\usepackage{subfig}
\usepackage{graphicx} % Allows including images
\usepackage{booktabs} % Allows the use of \toprule, \midrule and \bottomrule in tables
\definecolor{olivegreen}{RGB}{34,139,34}
\definecolor{darkorange}{RGB}{255,140,0}
%----------------------------------------------------------------------------------------
%	TITLE PAGE
%----------------------------------------------------------------------------------------

\title[CR proton from $\gamma$-ray Earth's Limb]{Preliminary indirect measurement of cosmic-ray proton spectrum using Earth's $\gamma$-ray data from {\it Fermi} Large Area Telescope }
% \logo{\includegraphics[height=0.75cm]{santa}}
\author[P. Payoungkhamdee]{
  Patomporn Payoungkhamdee$^{\dagger}$ \\
  Asst. Prof. Warit Mitthumsiri$^{\dagger}$, Prof. David Ruffolo$^{\dagger}$
} % Your name
\institute[MU] % Your institution as it will appear on the bottom of every slide, may be shorthand to save space
{
  $^{\dagger}$Department of Physics, Faculty of Science, Mahidol University \\ % Your institution for the title page
\medskip
\textit{patomporn.pay@gmail.com} % Your email address

}
\date[June 2020]{Siam Physics Congress 2020 \\ June 2020} % Date, can be changed to a custom date

\newcommand{\nologo}{\setbeamertemplate{logo}{}}

\begin{document}

{\nologo
\begin{frame}
\titlepage % Print the title page as the first slide
\end{frame}
}

\begin{frame}
\frametitle{Overview} % Table of contents slide, comment this block out to remove it
\tableofcontents % Throughout your presentation, if you choose to use \section{} and \subsection{} commands, these will automatically be printed on this slide as an overview of your presentation
\end{frame}

%----------------------------------------------------------------------------------------
%	PRESENTATION SLIDES
%----------------------------------------------------------------------------------------


%------------------------------------------------
%    1 ) Background
%------------------------------------------------
\section{Introduction}
% \section{Background}
\subsection{Background}
%---- 1.1 ) What is CRs ---
\begin{frame}
\frametitle{What are CRs}
  \begin{columns}
      \column{0.5\textwidth}

      \begin{figure}
      \includegraphics[height=0.7\textheight, width=0.9\textwidth]{CRFeature}
      \caption{CR spectral: figure from  universe-review.ca }
      \end{figure}

      \column{0.5\textwidth}
      \begin{itemize}
      \item{High energy particles in space}
      \\ 
      \item{\textbf{Criteria :} Here "flux" means differential flux}
      \\ 
      \item{\textbf{Feature :} CR rigidity spectrum can be described well by power-law }
      \item{Changes of power-law indices may come from superposition of different acceleration mechanisms}
      \end{itemize}
  \end{columns}
\end{frame}
%---- Intro objective (PAMELA & AMS) ---
\begin{frame}
\frametitle{Previous study}
\begin{itemize}
  \item In 2011, PAMELA claimed to discover a break in CR proton spectrum at around 300 GV.
  \item In 2014, \textit{Fermi} LAT found some hint of this break though the results were inconclusive.
  \item In 2015, the AMS-02 comfirmed this break.
\end{itemize}
% In 2015, the AMS collaboration claims that there is a broken in cosmic ray proton spectrum around 336 GV.
\begin{figure}
  \includegraphics[width=0.55\textwidth]{proton_spectrum}
  \caption{CR proton flux from Aguilar, et al., (2015)}
\end{figure}
\end{frame}

%------------------------------------------------
%      2 ) Objectives
%------------------------------------------------
\subsection{Objectives}
% \section{Objectives} % Sections can be created in order to organize your presentation into discrete blocks, all sections and subsections are automatically printed in the table of contents as an overview of the talk

%---- our point ---
\begin{frame}
\frametitle{Objective}
\begin{itemize}
  \item To measure CR proton spectrum between 60 GV - 2 TV using
  Earth's $\gamma$-ray data from \textit{Fermi}-LAT through Kachelrie$\beta$ and Ostapchenko model
  \item To test if we can use Fermi LAT data to confirm the spectral break at around 340 GV as observed by some experiments
\end{itemize}
\end{frame}

%------     Production schematics --------------
\subsection{Schematics of the Earth's $\gamma$-ray production}
\begin{frame}
\frametitle{Schematics of the Earth's $\gamma$-ray production}
\centering
\begin{figure}[h!]
\includegraphics[width = 0.8\textwidth]{lat_production_schematic}
\end{figure}
\end{frame}


%------------------------------------------------
%    3 ) Flux extraction
%------------------------------------------------
\section{Flux extraction}
%------------------------------------------------
%    3.2 ) Data set
%------------------------------------------------
\subsection{Data set}

\begin{frame}
\frametitle{Data selection}
\begin{itemize}
  \item P8R2\_ULTRACLEANVETO\_V6 data from 07/08/2008 to 17/10/2017 ($\sim$9 years) %Use P8R2\_ULTRACLEANVETO\_V6 data which collect only photon
  \item Photon energy range from 10 GeV up to 1 TeV
  \item $\theta_{\text{NADIR}}$ $\in$ 68.4$^\circ$  - 70$^\circ$ (Earth's limb)
  \item Use $\theta_{\text{LAT}} < 70^\circ$
\end{itemize}
\end{frame}
%------------------------------------------------
%    3.3 ) Flux calculation
%------------------------------------------------
\subsection{Flux calculation}
%--- How to calculate flux ---
\begin{frame}
\frametitle{Flux calculation method}
\begin{enumerate}
  \item Make 2D histograms with 25 bins per decade of energy
  \item Select photon data and fill in the 2D histograms
  \item Calculate exposure maps which include the effective area and livetime of the LAT as it observed the Earth
  \begin{equation}
    \textbf{Flux} \equiv \frac{dN_\gamma}{dE} = \frac{\int_{\text{Limb region}}(\text{Count map}/\text{Exposure map})}{\Delta\Omega\Delta E }
  \end{equation}
  % \item 
  % \item Sum over limb region of this map then divided by solidangle and energy bin width
  % \item Now we got $\gamma$-ray flux
\end{enumerate}
where $\Delta\Omega$ is the solid angle of the Earth's limb region and $\Delta E$ is the energy bin width
\end{frame}
% %--- cntmap ---
% \begin{frame}
% \frametitle{Count map}

% %\begin{figure}[h!]
% %\includegraphics[width = 0.9\textwidth]{cntmap}
% %\caption{Bla1}
% %\end{figure}

% \begin{figure}[h!]
% \begin{tikzpicture}
% \node (0,0) {\includegraphics[width=0.9\textwidth]{cntmap}};
% \node [opacity=0.2] (0,0) {\rotatebox{45}{\scalebox{3.5}{\textcolor{red}{preliminary}}}};
% \end{tikzpicture}
% \end{figure}

% \end{frame}
% %--- expmap ---
% \begin{frame}
% \frametitle{Exposure map}

% %\begin{figure}[h!]
% %\includegraphics[width = 0.9\textwidth]{expmap}
% %\caption{Bla1}
% %\end{figure}

% \begin{figure}[h!]
% \begin{tikzpicture}
% \node (0,0) {\includegraphics[width=0.9\textwidth]{expmap}};
% \node [opacity=0.2] (0,0) {\rotatebox{45}{\scalebox{3.5}{\textcolor{red}{preliminary}}}};
% \end{tikzpicture}
% \end{figure}

% \end{frame}

% %--- flxmap ---
% \begin{frame}
% \frametitle{Flux map}

% %\begin{figure}[h!]
% %\includegraphics[width = 0.9\textwidth]{flxmap}
% %\caption{Bla1}
% %\end{figure}

% \begin{figure}[h!]
% \begin{tikzpicture}
% \node (0,0) {\includegraphics[width=0.9\textwidth]{flxmap}};
% \node [opacity=0.2] (0,0) {\rotatebox{45}{\scalebox{3.5}{\textcolor{red}{preliminary}}}};
% \end{tikzpicture}
% \end{figure}


%--- got spectrum ---
\begin{frame}
\frametitle{Earth's limb $\gamma$-ray spectrum from measurement}

%\begin{figure}[h!]
%\includegraphics[width = 0.7\textwidth]{FluxfromExposure}
%\end{figure}

\begin{figure}[h!]
\begin{tikzpicture}
\node (0,0) {\includegraphics[width=0.6\textwidth]{figure/gamma_spectrum}};
\node [opacity=0.2] (0,0) {\rotatebox{45}{\scalebox{2.5}{\textcolor{red}{preliminary}}}};
% \node [opacity=1.0] at (2.4,-2.40) {\tiny E (GeV)};
\end{tikzpicture}
\end{figure}
\begin{itemize}
  \item Error bars show statistical uncertainties and red bands show total (statistical + systematic) uncertainties
  \item The amount of data in this work is about 2 times greater than previously published analysis by the LAT
\end{itemize}
\end{frame}

%------------------------------------------------
%    4 ) Analysis
%------------------------------------------------
\section{Analysis}

% ------- poewr law in rigidity -------------
\subsection{Power-law spectrum}
\begin{frame}
  \frametitle{Power-law models (in rigidity)}
  We use 2 models of CR proton to fit the $\gamma$-ray data: \\
  \textbf{Single power law (SPL)}
  \begin{equation}
  \frac{dN}{dR} = R_0R^{-\gamma}
  \end{equation}
  \textbf{Broken power law (BPL)}
  \begin{equation}
  \frac{dN}{dR}=
    \begin{cases}
      R_0R^{-\gamma_1}\ :\ E < E_{\text{Break}}\\
      R_0[R(E_{\text{Break}})]^{\gamma_2-\gamma_1}R^{-\gamma_2}\ :\ E \ge E_{\text{Break}}
    \end{cases}
  \end{equation}
  Rigidity is defined by $R\equiv P/q$ where $P$ is the momentum and $q$ is the absolute value of the charge (in unit of proton charge) of a particle
  \end{frame}
%------------------------------------------------
%    4.1 ) K&O model
%------------------------------------------------
\subsection[K$\&$O model]{Kachelrie$\beta$ and Ostapchenko proton-proton $\rightarrow$ $\gamma$ model}
\begin{frame}
\frametitle{Kachelrie$\beta$ and Ostapchenko model}
This model can compute the $\gamma$-ray spectrum from a broad and smooth power-law spectrum of CR protons
% Is the model which can compute spectrum of $\gamma$-ray from a known incident proton
\begin{equation}
  % \small
  \frac{dN_{\gamma}}{dE} \propto \sum_{E_{\text{inc,i}}}\left[\frac{E_{\text{inc,i}}}{E_{\gamma\text{,i}}}\Delta(E_{\text{inc,i}}) \right]\left[ f_{pp}\textcolor{red}{\frac{dN_\text{H}}{dE_{\text{inc,i}}}}\left\{ 1+\textcolor{olivegreen}{\frac{\sigma_{\text{HeN}}}{\sigma{pN}}}\left(\textcolor{red}{\frac{dN_{\text{H}}}{dR}}\right)^{-1} \textcolor{blue}{\frac{dN_{\text{He}}}{dR}} \frac{dR_{\text{He}}}{dR_{\text{H}}}  \right\}\right]
\end{equation}
\begin{itemize}
  \item Red color terms are for \textcolor{red}{incident proton spectrum}
  \item \textcolor{blue}{Blue color term is the He spectrum from AMS-02 (2015)}
  \item $f_{pp}\equiv E_\gamma(d\sigma^{pp\rightarrow\gamma}/dE_\gamma)$ is the interaction cross section table in the K\&O model
  \item The cross-section ratio \textcolor{olivegreen}{$\sigma_{\text{HeN}}/\sigma_{pN}$} at high energy ($>$ 10GeV) is roughly constant ($\approx 1.6$)\footnote{T. W. Atwater (2015)}
\end{itemize}

\end{frame}
%------------------------------------------------
%    4.1 Optimization
%------------------------------------------------
\subsection{Optimization}
\begin{frame}
\frametitle{Poisson likelihood function}
% On the previous slide, we want to find the incident proton. \\
% Let define some loss function to compare model and measurement
% \begin{equation}
%   \mathcal{L} = \prod_{i=1}^{N} P_{\text{pois}}(n_{\text{i,model}}, n_{\text{i,measurement}})
% \end{equation}
We determine the incident proton spectrum that best fits the $\gamma$-ray masurement using the maximum likelihood (or minimum log likelihood) method

% For numerically convenient, redefined into logarithmic form
\begin{equation}
  \log\mathcal{L} \equiv \sum_{i=1}^{\textcolor{blue}{N}} -\log P_{\text{pois}}(n_{\text{i,model}}, n_{\text{i,measurement}})
\end{equation}
% This part is the hard work of computer to find best incident cosmic ray proton that match the
% spectrum from measurement.
where $P_{\text{pois}}$ is the Poisson probability of measuring $n_{\text{i,measurement}}$ counts when the model predicts $n_{\text{i,model}}$ counts for \textit{\textcolor{blue}{N}} energy bins

\end{frame}

\begin{frame}\frametitle{Fitting algorithm: Particle Swarm Optimization}
\begin{itemize}
  \item Randomly initiate many particles in a given range of the parameter space
  \item Check global and local best particle from a defined profit function
  \item Rest of them move toward the global and local particles
  \item Iterate the process until most of them yield nearly the same profit
\end{itemize}
\begin{figure}
\begin{columns}
  \begin{column}{0.34\textwidth}
    \includegraphics[width=\columnwidth]{figure/particle_swarm0002}
  \end{column}
  \begin{column}{0.34\textwidth}
  \includegraphics[width=\columnwidth]{figure/particle_swarm0011}
  \end{column}
  \begin{column}{0.34\textwidth}
    \includegraphics[width=\columnwidth]{figure/particle_swarm0061}
  \end{column}
\end{columns}
\caption{Example of particles in parameter space of Ackley potential}
\end{figure}
\end{frame}


\begin{frame}\frametitle{Particle Swarm Optimization}
For every iteration $k$, particle $i$ move with velocity $v^i_k$ where
\begin{equation}
  v^i_{k+1} = \omega v^i_k + c^br^b_k[b^i_k-x^i_k] + c^Br^B_k[B^i_k-x^i_k]
\end{equation}
Update the new state of particle $i$ with
\begin{equation}
  x^i_{k+1} = x^i_k + v^i_{k+1}
\end{equation}
where
\begin{itemize}
  \item $x^i_k$ represent variable that particle $i$ hold
  \item $b$ and $B$ are best local and global parameter sets along the optimization process
  \item Set $\omega = 0.2$, $c^b = 0.2$ and $c^B = 0.3$
\end{itemize}
The iteration process would stop when standard deviation of fitness over any partilcle less than 0.1
\end{frame}

% %-- Flow chart ---%
%   % Define block styles
%   \tikzstyle{decision} = [diamond, draw, fill=red!10, 
%   text width=2em, text badly centered, node distance=2.5cm, inner sep=0pt]
%   \tikzstyle{block} = [rectangle, draw, fill=blue!10, 
%   text width=5em, text centered, rounded corners, minimum height=2em]
%   \tikzstyle{longblock} = [rectangle, draw, fill=blue!10, 
%   text width=10em, text centered, rounded corners, minimum height=4em]
%   \tikzstyle{line} = [draw,thick, -latex']
%   \tikzstyle{cloud} = [draw, ellipse,fill=red!20, node distance=3cm,
%   minimum height=4em]

% \begin{frame}\frametitle{Algorithm}

% \begin{figure}[!h]
% \centering
% \begin{tikzpicture}[node distance = 6em, auto]
%     \node [block, fill=black!10] (powerlaw) {powerlaw spectrum of proton in rigidity};
%     \node [block, right of = powerlaw, node distance = 12em] (model) {$\gamma$-ray spectrum from model};
%     \node [block, right of = model] (measurement) {$\gamma$-ray spectrum from measurement};
%     \node [block, below left of = measurement, node distance = 8em] (pois) {Poisson likelihood function};
%     \node [block, below of = powerlaw] (varyPar) {Apply gradient to parameters $\gamma$, $R_0$};
%     \node [decision, below of = varyPar, fill = blue!10] (bestfit) {Best fit?};
%     \node [block, right of = bestfit, node distance = 10em, fill = red!10] (return) {Return best fit parameters};

%     \path [line] (powerlaw) -- node [anchor=south] {K\&O model} (model);
%     \path [line] (model) -- (pois);
%     \path [line] (measurement) -- (pois);
%     \path [line] (pois) -- (bestfit);
%     \path [line] (bestfit) -- node [anchor=east] {No} (varyPar);
%     \path [line] (varyPar) -- (powerlaw);
%     \path [line] (bestfit) -- node [anchor=north] {Yes} (return);
% \end{tikzpicture}
% \caption{Flow chart of optimization process}
% \end{figure}

% \end{frame}

%------------------------------------------------
%    4.3 ) Results
%------------------------------------------------
\section{Results}
\begin{frame}
\frametitle{Results}
\begin{table}
\begin{tabular}{l | c | c | c}
  Best fits & $\gamma_1$ & $\gamma_2$ & $E_{\text{Break}}$ (GeV) \\
  \hline \hline
  SPL & 2.70 & - & -  \\
  BPL & 2.86  & 2.63 & 333
\end{tabular}
\caption{Optimization results}
\end{table}

\end{frame}

% --- graph gamma ray
\begin{frame}
\frametitle{Earth's limb $\gamma$-ray spectra from best-fit models}

\begin{figure}
  \begin{tikzpicture}
  \node (0,0) {\includegraphics[width=0.85\textwidth]{figure/fitted_result}};
  \node [opacity=0.2] (0,0) {\rotatebox{45}{\scalebox{2.5}{\textcolor{red}{preliminary}}}};
  % \node [opacity=1.0] at (2.7,-2.75) {\tiny E (GeV)};
  \end{tikzpicture}
\end{figure}


\end{frame}
% --- spectrum proton
\begin{frame}
  \frametitle{Proton spectrum}
  
  \begin{figure}[h!]
  \begin{tikzpicture}
  \node (0,0) {\includegraphics[width=0.8\textwidth]{ProtonSpectrumModelMeasurement}};
  \node [opacity=0.2] (0,0) {\rotatebox{45}{\scalebox{2.5}{\textcolor{red}{preliminary}}}};
  \end{tikzpicture}
  \end{figure}
\tiny The normalization of this work is fitted PAMELA data  
\end{frame}
%------------------------------------------------
%    4.4 ) Monte carlo Simulation
%------------------------------------------------
% \subsection{Monte Carlo simlation}
% %----- How to deal with this ?
% \begin{frame}
% \frametitle{Error determination}
% \textbf{Statictical error (Random error)}
% \begin{enumerate}
%   \item Get back to raw count and
%   \textcolor{blue}{random new count in each energy bin by Poisson random function}
%   \item Recalculate proton spectrum
%   \item Optimize it and store the parameter that we got
%   \item do it over thoundsand time and fill in histogram to interpret error by saying sigma of gaussian function
% \end{enumerate}
% \textbf{Total error (take into account instrument)}
% \begin{enumerate}
%   \item Get back to raw count and
%   \textcolor{blue}{random new count in each energy bin by Poisson random function}
%   \item \textcolor{blue}{Random value we got again by systematic error (Apparatus)}
%   \item Recalculate proton spectrum
%   \item Optimize it and store the parameter that we got
%   \item do it over thoundsand time and fill in histogram to interpret error by saying sigma of gaussian function
% \end{enumerate}
% \end{frame}

% %----- SPL ------
% \begin{frame}
% \frametitle{Single power law (SPL)}

% \begin{figure}[h!]
% \begin{tikzpicture}
% \node (0,0) {\includegraphics[width=0.8\textwidth]{SPL_Sim}};
% \node [opacity=0.2] (0,0) {\rotatebox{45}{\scalebox{3.0}{\textcolor{red}{preliminary}}}};
% \end{tikzpicture}
% \end{figure}

% \end{frame}
% %------ BPL ------
% \begin{frame}
% \frametitle{Broken power law (BPL)}

% %\begin{figure}[h!]
% %\includegraphics[width = \textwidth]{BPL_Sim}
% %\end{figure}

% \begin{figure}[h!]
% \begin{tikzpicture}
% \node (0,0) {\includegraphics[width=\textwidth]{BPL_Sim}};
% \node [opacity=0.2] (0,0) {\rotatebox{25}{\scalebox{3.5}{\textcolor{red}{preliminary}}}};
% \end{tikzpicture}
% \end{figure}

% \end{frame}
% %------------------------------------------------
% %   4 )  To do list
% %------------------------------------------------
% \section{To do list}
% \begin{frame}
% \frametitle{To do}
% \begin{itemize}
% \item Problem of shifted value in BPL came from optimization algorithm because it stop in local minimum
% \item Extract new Flux due to technical problem (Still doubt.. might came from exposure map or photon selection)
% \item Fix bug in my log-likelihood function (Use count instread of flux)
% \item Resimulate data when I am done above stuff
% \end{itemize}
% \end{frame}

%------------------------------------------------
%   Conclusion
%------------------------------------------------
\section{Future work}
\begin{frame}
\frametitle{Future work}
\begin{itemize}
  \item Calculate the errors of the fitted parameters for both the SPL and BPL models of CR proton spectrum
  \item Determine if the BPL model of CR proton spectrum fits the $\gamma$-ray data significantly better than the SPL model does
\end{itemize}
\end{frame}


%------------------------------------------------


%------------------------------------------------
\section{} % just empty section for not showing header on slide

%------------------------------------------------
%    Reference
%------------------------------------------------
\section{Reference}
\begin{frame}
\frametitle{References}
[1] O. Adriani et al., Science 332, 69 (2011) \newline
[2] M. Ackermann et al. (\textit{Fermi} LAT Collaboration), Phys. Rev. Lett. 112, 151103 (2015) \newline
[3] Kachelriess $\&$ Ostapchenko, Phys. Rev. D 86 (2012) \newline
[4] M. Aguilar et al. (AMS Collaboration), Phys. Rev. Lett. 115, 211101 (2015) \newline
[5] M. Aguilar et al. (AMS Collaboration), Phys. Rev. Lett. 114, 171103 (2015) \newline
% [6] L. Lyons, Statistics for nuclear and particle physicists
\end{frame}
%------------------------------------------------
%    Acknowledgement
%------------------------------------------------
\section{}
\begin{frame}\frametitle{Acknowledgement}
  \begin{itemize}
    % \item Dr. Warit Mitthumsiri \\ Mahidol University, Thailand
    \item Dr. Francesca Spada \\ University of Pisa, Italy
    \item People in the Space Physics Laboratory at Mahidol University and the \textit{Fermi}-LAT research group at the University of Pisa
    \item Development and Promotion of Science and Technology Talents Project (DPST)
    \item Partially supported by the Thailand Science Research and Innovation (RTA6280002)
  \end{itemize}
\end{frame}

%------------------------------------------------
%    Back up slide
%------------------------------------------------

\begin{frame}
\Huge{\centerline{Backup slide}}
\end{frame}

% % ------- poewr law in rigidity -------------
% \begin{frame}
% \frametitle{Power law (in rigidity)}
% Typically, cosmic ray spectrum follow power law in rigidity as \\
% \textbf{Single power law (SPL)}
% \begin{equation}
% \frac{dN}{dR} = R_0R^{-\gamma}
% \end{equation}
% \textbf{Broken power law (BPL)}
% \begin{equation}
% \frac{dN}{dR}=
%   \begin{cases}
%     R_0R^{-\gamma_1}\ :\ E < E_{\text{Break}}\\
%     R_0[R(E_{\text{Break}})]^{\gamma_2-\gamma_1}R^{-\gamma_2}\ :\ E \ge E_{\text{Break}}
%   \end{cases}
% \end{equation}
% Note for someone who not famoliar with rigidity : it just defined by $R\equiv P/q$ when $P, q$ is a momentum and charge of particle
% \end{frame}
% ------- power law in Energy -------------
\begin{frame}
\frametitle{Power law in energy}
Converting the power law in rigidity to energy, we obtain
\textbf{Single power law (SPL)}
\begin{equation}
\frac{dN}{dE} = N_0[E_k(E_k+2m_p)]^{-\gamma/2} \left(\frac{E_k+m_p}{\sqrt{E_k(E_k+2m_p)}}\right)
\end{equation}
\textbf{Broken power law (BPL)}
\begin{equation}
\frac{dN}{dE}=
  \begin{cases}
    N_0[E_k(E_k+2m_p)]^{-\gamma_1/2} \left(\frac{E_k+m_p}{\sqrt{E_k(E_k+2m_p)}}\right)\ :\ E < E_{\text{Break}}\\
    N_0[E_b(E_b+2m_p)]^{(\gamma_2-\gamma_1)/2}[E_k(E_k+2m_p)]^{-\gamma_2/2} \left(\frac{E_k+m_p}{\sqrt{E_k(E_k+2m_p)}}\right)\\ :\ E \ge E_{\text{Break}}
  \end{cases}
\end{equation}
\end{frame}


%----------------------------------
% ---------  calculation map --------
%----------------------------------
%--- cntmap ---
\begin{frame}
\frametitle{Count map}
\begin{figure}[h!]
\begin{tikzpicture}
\node (0,0) {\includegraphics[height=0.9\textheight]{figure/cartesian_cntmaps}};
\node [opacity=0.2] (0,0) {\rotatebox{45}{\scalebox{3.5}{\textcolor{red}{preliminary}}}};
\end{tikzpicture}
\end{figure}
\end{frame}

\begin{frame}
\frametitle{Count maps}
\begin{figure}[h!]
\begin{tikzpicture}
\node (0,0) {\includegraphics[height=0.9\textheight]{figure/polar_cntmaps}};
\node [opacity=0.2] (0,0) {\rotatebox{45}{\scalebox{3.5}{\textcolor{red}{preliminary}}}};
\end{tikzpicture}
\end{figure}
\end{frame}
%--- expmap ---
\begin{frame}
\frametitle{Exposure maps}
\begin{figure}[h!]
\begin{tikzpicture}
\node (0,0) {\includegraphics[height=0.9\textheight]{figure/cartesian_expmaps}};
\node [opacity=0.2] (0,0) {\rotatebox{45}{\scalebox{3.5}{\textcolor{red}{preliminary}}}};
\end{tikzpicture}
\end{figure}
\end{frame}

\begin{frame}
\frametitle{Exposure maps}
\begin{figure}[h!]
\begin{tikzpicture}
\node (0,0) {\includegraphics[height=0.9\textheight]{figure/polar_expmaps}};
\node [opacity=0.2] (0,0) {\rotatebox{45}{\scalebox{3.5}{\textcolor{red}{preliminary}}}};
\end{tikzpicture}
\end{figure}
\end{frame}
%--- flxmap ---
\begin{frame}
\frametitle{Flux maps}
\begin{figure}[h!]
\begin{tikzpicture}
\node (0,0) {\includegraphics[height=0.9\textheight]{figure/cartesian_flxmaps}};
\node [opacity=0.2] (0,0) {\rotatebox{45}{\scalebox{3.5}{\textcolor{red}{preliminary}}}};
\end{tikzpicture}
\end{figure}
\end{frame}

\begin{frame}
\frametitle{Flux maps}
\begin{figure}[h!]
\begin{tikzpicture}
\node (0,0) {\includegraphics[height=0.9\textheight]{figure/polar_flxmaps}};
\node [opacity=0.2] (0,0) {\rotatebox{45}{\scalebox{3.5}{\textcolor{red}{preliminary}}}};
\end{tikzpicture}
\end{figure}
\end{frame}


%----------------------------------------------------------------------------------------

\end{document}
