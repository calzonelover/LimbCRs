\par Cosmic-ray research has been pioneered by Theodor Wulf who taked electrometer measured cosmic ray from the ground to higher altitude and much more experiment has confirmed that there is a cosmic ray from outer space which can penetrate and interect with the Earth atmosphere \cite{HESS,Pacini,Clay}.

\par There are many possible phenomena of acceleration mechanism in the
space that could produced high energy particles. Consequently, characteristic of acceleration machanism could roughly be distinguished by a spectral index in the arrival of cosmic ray spectrum in rigidity.
Breaking point of the spectrum mainly come from the overlapped region of acceleration mechanism that could be an evidence to explore a new candidate of acceleration type.

In 2011, Pamela detecter indicate that there is a break point of proton cosmic ray spectrum around 240 GV \cite{PAMELA}.
Furthermore, AMS-02 also found a drastic change of proton CR spectrum at around 336 GV \cite{AMS-02}.

\par In this work, the indirect measurement of proton cosmic ray will be perform by using gamma-ray data from \textit{Fermi} Large Area Telescope (\textit{Fermi}-LAT).